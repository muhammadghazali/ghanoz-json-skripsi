% Laporan Tugas Akhir
% Muhammad Ghazali - 0606036
\documentclass[a4paper, 12pt]{report}
\usepackage{graphicx, times}
\usepackage[bahasa]{babel}
\usepackage{tabularx}

\selectlanguage{bahasa}
%Gummi|063|=)
\title{\textbf{Membangun\ RESTful\ Web Service\ dengan\ Menggunakan\ NodeJS}}
\author{
Muhammad Ghazali\\
Program Studi Teknik Informatika\\
Fakultas Teknik\\
Universitas Widyatama
\\ \texttt{<muhammadghazali2480@gmail.com>}
}
\date{\today}

\begin{document}

\maketitle

\tableofcontents
\setcounter{tocdepth}{3}

\listoffigures
\listoftables

% most research papers have an abstract, then there is a predefined commands
% for telling LaTeX which part of the content makes up the abstract. This
% should appear in its logical order, therefore, after the top matter, but
% before the main sections of the body.
\begin{abstract}
Abstract lorem ipsum is simply dummy text of the printing and typesetting industry. Lorem Ipsum has been the industry's standard dummy text ever since the 1500s, when an unknown printer took a galley of type and scrambled it to make a type specimen book. It has survived not only five centuries, but also the leap into electronic typesetting, remaining essentially unchanged. It was popularised in the 1960s with the release of Letraset sheets containing Lorem Ipsum passages, and more recently with desktop publishing software like Aldus PageMaker including versions of Lorem Ipsum.
\end{abstract}

\section*{Latar Belakang dan Masalah}
\addcontentsline{toc}{subsection}{Latar Belakang dan Masalah}
\begin{flushleft}
TODO Jelaskan latar belakang masalah
\end{flushleft}

\section*{Rumusan Masalah}
\addcontentsline{toc}{subsection}{Rumusan Masalah}
\begin{flushleft}
TODO Jelaskan rumusan masalah
\end{flushleft}

\section*{Batasan Masalah}
\addcontentsline{toc}{subsection}{Batasan Masalah}
\begin{flushleft}
TODO Jelaskan batasan masalah
\end{flushleft}

\section*{Tujuan}
\addcontentsline{toc}{subsection}{Tujuan}
\begin{flushleft}
Jelaskan tujuan penelitian
\end{flushleft}

\section*{Manfaat Penelitian}
\addcontentsline{toc}{subsection}{Manfaat Penelitian}
\begin{flushleft}
Jelaskan manfaat penelitian untuk LayangLayang Mobile dan para pengembang perangkat lunak lain?
\end{flushleft}

\section*{Landasan Teori}
\addcontentsline{toc}{subsection}{Landasan Teori}
\begin{flushleft}
TODO Jelaskan teori-teori yang mendasari penelitian ini.
\end{flushleft}

\section*{Metodologi Penelitian}
\addcontentsline{toc}{subsection}{Metodologi Penelitian}
\begin{flushleft}
TODO Jelaskan Metodologi Penelitian
TODO Jelaskan Prosedur Penelitian
TODO Hipotesa
\end{flushleft}

\section*{Jadwal Kerja}
\addcontentsline{toc}{subsection}{Jadwal Kerja}
\begin{flushleft}
TODO Jadwal Kerja
\end{flushleft}

\section*{Usulan Pembimbing}
\addcontentsline{toc}{subsection}{Usulan Pembimbing}
Saya mengharapkan pembimbing yang benar-benar paham di bidang rekayasa perangkat lunak dan saya mengusulkan ibu Sriyani Violina untuk menjadi pembimbing pengerjaan tugas akhir saya.\cite{lamport94}

% Daftar buku atau karangan yang merupakan sumber rujukan dari sebuah
% tulisan atau karangan atau daftar tt suatu subjek ilmu, daftar pustaka
% Jumlah maksimal daftar pustaka dan referensi yang bisa dimasukkan adalah 100 item
\begin{thebibliography}{99}

% use the surname of the first author, followed by the last two digits of
% the year (hence lamport94)
\bibitem{lamport94}
  Leslie Lamport,
  \emph{\LaTeX: A Document Preparation System}.
  Addison Wesley, Massachusetts,
  2nd Edition,
  1994.

\bibitem{lamport95}
  Leslie Lamport,
  \emph{\LaTeX: A Document Preparation System}.
  Addison Wesley, Massachusetts,
  2nd Edition,
  1995.

\end{thebibliography}

\end{document}

