% Proposal Skripsi
% Muhammad Ghazali - 0606036
\documentclass[a4paper, 12pt]{report}
\usepackage{setspace}
\usepackage{graphicx, times}
\usepackage[bahasa]{babel}
\usepackage{tikz}
\usepackage{gantt}
\usepackage{tabularx}
\usepackage[top=3cm, bottom=3cm, left=4cm, right=3cm]{geometry}

\selectlanguage{bahasa}
%Gummi|063|=)
\title{\textbf{Membangun Web API dengan menggunakan JSON sebagai format serialisasi data}}
\author{
Muhammad Ghazali\\
Program Studi Teknik Informatika\\
Fakultas Teknik\\
Universitas Widyatama
\\\texttt{<muhammad.ghazali@widyatama.ac.id>}
}
\date{\today}


\begin{document}

\maketitle

\onehalfspacing
\tableofcontents
\setcounter{tocdepth}{3}

% most research papers have an abstract, then there is a predefined commands
% for telling LaTeX which part of the content makes up the abstract. This
% should appear in its logical order, therefore, after the top matter, but
% before the main sections of the body.
\begin{abstract}
\onehalfspacing LayangLayang Mobile (LLM) merupakan salah perusahaan yang bergerak di bidang \textit{mobile application development}. Saat ini LayangLayang Mobile sedang mengembangkan sebuah produk bernama CampusLife. Produk yang dikembangkan tersebut merupakan aplikasi \textit{mobile} yang bertujuan untuk membantu civitas kampus mengakses informasi relevan tentang kampus mereka.

\onehalfspacing Setiap informasi yang ditampilkan melalui aplikasi \textit{mobile} CampusLife merupakan data yang sudah diolah dan diambil dari Web API\footnote{http://en.wikipedia.org/wiki/Web\_API} CampusLife. Saat ini LLM belum memiliki Web API tersebut. Berdasarkan kondisi tersebut, penulis bekerjasama dengan LLM untuk membangun Web API CampusLife. Web API yang akan dibangun bertujuan untuk membuka akses secara tidak langsung ke \textit{data store}\footnote{http://en.wikipedia.org/wiki/Data\_store} yang tersimpan di salah satu layanan \textit{Database as a Service}\footnote{http://en.wikipedia.org/wiki/Cloud\_database} yang digunakan oleh LLM di AppFog\footnote{http://www.appfog.com/}. Seluruh data-data \textit{event} yang tersimpan di \textit{data store} akan diolah oleh Web API menjadi data dengan format yang dapat dikonsumsi dengan mudah oleh aplikasi \textit{mobile} CampusLife. Proses pengelohan tersebut dinamakan serialisasi data\footnote{Lihat bagian Landasan teori: Serialiasi Data}.

\onehalfspacing Dalam penelitian ini penulis akan memilih format serialisasi data JSON untuk digunakan merepresentasikan setiap data-data \textit{event} dalam format yang dapat dikonsumsi oleh aplikasi \textit{mobile} CampusLife. Penulis memilih format serialisasi data JSON karena JSON lebih mudah dibaca ditulis dan dibaca oleh mesin (komputer) dan manusia. Selain itu JSON lebih mudah untuk diproses karena memiliki struktur yang lebih sederhana dibandingkan XML\cite{json-fat-free}\cite{json-vs-xml-debate}.

\begin{flushleft}
\onehalfspacing Kata kunci: Web API, JSON, Format Serialisasi Data
\end{flushleft}

\end{abstract}

% isi latar belakang dan masalah dibuat dengan mengikuti panduan:
% http://romisatriawahono.net/2012/06/18/kiat-menyusun-alur-latar-belakang-masalah-penelitian/
\section*{Latar Belakang dan Masalah}
\addcontentsline{toc}{subsection}{Latar Belakang dan Masalah}
% menjelaskan obyek penelitian
\onehalfspacing CampusLife adalah \textit{mobile information directory application} yang dikembangkan oleh LayangLayang Mobile (LLM) untuk menyediakan informasi yang relevan kepada civitas kampus. Salah satu fitur utama yang akan dirilis dalam waktu dekat adalah menyediakan informasi \textit{event}-\textit{event} terbaru kepada civitas kampus. 

\onehalfspacing Setiap informasi yang ditampilkan melalui aplikasi \textit{mobile} CampusLife merupakan data yang sudah diolah dan diambil dari Web API\footnote{http://en.wikipedia.org/wiki/Web\_API} CampusLife. Saat ini LLM belum memiliki Web API dan penulis berniat untuk membangun Web API tersebut. Web API yang akan dibangun bertujuan untuk membuka akses secara tidak langsung ke \textit{data store}\footnote{http://en.wikipedia.org/wiki/Data\_store} yang tersimpan di salah satu layanan \textit{Database as a Service}\footnote{http://en.wikipedia.org/wiki/Cloud\_database} yang digunakan oleh LLM di AppFog\footnote{http://www.appfog.com/}. Seluruh data-data \textit{event} yang tersimpan di \textit{data store} akan diolah oleh Web API menjadi data dengan format yang dapat dikonsumsi dengan mudah oleh aplikasi \textit{mobile} CampusLife. Proses pengolahan tersebut dinamakan serialisasi data\footnote{Lihat bagian Landasan teori: Serialiasi Data}.

\onehalfspacing Sampai awal tahun 2013 ini, sudah ada lebih dari 10 format serialisasi data\footnote{http://en.wikipedia.org/wiki/Comparison\_of\_data\_serialization\_formats}:
\begin{enumerate}
  \item ASN.1
  \item Bencode
  \item \textit{Candle Markup}
  \item \textit{Comma-separated values} (CSV)
  \item BSON
  \item \textit{D-Bus Message Protocol}
  \item JSON
  \item MessagePack
  \item Netstrings
  \item OGDL
  \item \textit{Property list}
  \item \textit{Protocol Buffers}
  \item S-expressions
  \item Sereal
  \item \textit{Structured Data eXchange Formats}
  \item Thrift
  \item \textit{eXternal Data Representation}
  \item XML
  \item XML-RPC
  \item YAML
\end{enumerate}

\onehalfspacing Di antara format serialisasi data yang sudah disebutkan di atas, XML\footnote{http://www.w3.org/XML/} dan JSON\footnote{http://json.org/} merupakan format serialisasi data yang paling terkenal saat ini\cite{comparison-of-data-serialization-formats}. Dalam penelitian ini penulis akan memilih format serialisasi data JSON untuk digunakan merepresentasikan setiap data-data \textit{event} dalam format yang dapat dikonsumsi oleh aplikasi \textit{mobile} CampusLife. Penulis memilih format serialisasi data JSON karena JSON lebih mudah dibaca ditulis dan dibaca oleh mesin (komputer) dan manusia. Selain itu JSON lebih mudah untuk diproses karena memiliki struktur yang lebih sederhana dibandingkan XML\cite{json-fat-free}\cite{json-vs-xml-debate}.

% menjelaskan rangkuman penelitian
\onehalfspacing \textit{Smartphone} sebagai perangkat tempat aplikasi \textit{mobile} CampusLife berjalan, merupakan salah satu \textit{mobile computing devices} yang memiliki masa hidup baterai dan ketersediaan \textit{bandwidth} yang terbatas.\cite{challenging-issues-and-limitations-of-mobile-computing}. Dengan kedua keterbatasan tersebut, dalam penelitian ini penulis akan mengkaji penerapan format serialisasi data JSON yang efektif untuk menghasilkan ukuran data yang optimal saat pengiriman data berlangsung dari Web API ke aplikasi \textit{mobile} CampusLife. Hasil akhir yang diharapkan adalah format serialisasi data JSON mampu menghasilkan ukuran data yang optimal untuk dikonsumsi oleh aplikasi \textit{mobile}.

% Pada bagian ini jelaskan rumusan masalah yang akan diteliti
\newpage
\section*{Rumusan Masalah}
\addcontentsline{toc}{subsection}{Rumusan Masalah}
\onehalfspacing Berdasarkan latar belakang masalah yang telah dipaparkan, maka masalah yang akan diteliti dalam penelitian ini dapat dirumuskan sebagai berikut:
\begin{enumerate}
  \item Bagaimana JSON dapat diterapkan secara efektif agar data yang dihasilkan dapat memiliki ukuran data yang optimal untuk dikonsumsi oleh aplikasi \textit{mobile} CampusLife?
\end{enumerate}

% Tujuan dan manfaat penelitian
\section*{Tujuan dan Manfaat Penelitian}
\addcontentsline{toc}{subsection}{Rumusan Masalah}
\onehalfspacing Tujuan dari pelaksanaan penelitian tugas akhir yang dilakukan penulis adalah membantu LLM untuk membangun purwa-rupa Web API yang mampu mengirimkan data dengan ukuran data yang optimal untuk dikonsumsi oleh aplikasi \textit{mobile} CampusLife.

% TODO sempurnakan
\onehalfspacing Penulis mengharapkan hasil peneletian ini akan membawa manfaat positif bagi kepentingan dunia akademik dan praktis dalam hal penerapan format serialisasi data JSON untuk menghasilkan ukuran data yang optimal untuk dikonsumsi oleh aplikasi \textit{mobile}.

\section*{Batasan Masalah}
\addcontentsline{toc}{subsection}{Batasan Masalah}
\onehalfspacing Untuk memperjelas ruang lingkup pelaksanaan penelitian, penulis memiliki batasan masalah meliputi:
\onehalfspacing
\begin{itemize}
  \item Pembangunan Web API hanya akan sampai pada tahap purwa-rupa.
  \item Pembangunan Web API hanya akan meliputi API untuk mengambil data-data \textit{event}.
  \item JSON hanya mampu merepresentasikan data dalam bentuk teks, oleh karena itu data yang akan digunakan hanya terbatas pada data yang berbasis teks.
  \item Skema data \textit{event} akan disediakan oleh pihak LLM.
  \item Penulis akan melakukan demo untuk mengakses Web API melalui Android\footnote{http://www.android.com/} \textit{smartphone} yang sudah terpasang aplikasi mobile CampusLife. Aktivitas demo akan difokuskan pada pengambilan  data-data \textit{event} dari purwa-rupa Web API yang dibuat oleh penulis.
  \item Tidak membahas mengenai keamanan perangkat lunak, data dan jaringan.
  \item Pengembangan perangkat lunak menggunakan sebagian praktek dari \textit{Agile} dan tidak akan membahas \textit{Agile} secara komprehensif.
\end{itemize}

% TODO verifikasi
\section*{Metodologi Penelitian}
\addcontentsline{toc}{subsection}{Metodologi Penelitian}

\onehalfspacing Adapun tahapan penelitian yang akan dilakukan oleh penulis tahap-tahap berikut:
\begin{enumerate}
  \item Identifikasi masalah
  \item Perumusan hipotesis
  \item Pengujian hipotesis
  \item Kesimpulan
\end{enumerate}

% TODO sempurnakan
\onehalfspacing Adapun cara untuk menunjang penelitian dilakukan dengan melakukan studi kepustakaan, yaitu pengumpulan data dari artikel-artikel, jurnal dan buku-buku yang berhubungan dengan topik pembahasan di penelitian.

% sistematika penulisan
\section*{Sistematika Penulisan}
\addcontentsline{toc}{subsection}{Prosedur Penelitian}
Sistematika pembahasan laporan ini terdiri dari enam bab, yaitu:
\begin{description}
  \item[Bab I Pendahuhuluan] Pada bagian ini berisikan pendahuluan laporan yang berisi latar belakang, rumusan masalah, tujuan, batasan masalah, metode penelitian dan sistematika penulisan laporan.
  \item[Bab II Landasan Teori] Pada bagian ini akan dibahas landasan teori yang berkaitan dan digunakan selama masa penelitian.
  \item[Bab III Analisis dan Desain] Pada bagian ini akan dibahas hasil analisa terhadap masalah dan rancangan solusi yang diterapkan.
  \item[Bab IV Konstruksi Perangkat Lunak] Pada bagian ini akan dibahas hasil konstruksi perangkat lunak sesuai dengan rancangan yang sudah ditentukan.
  \item[Bab V Pengujian] Pada bagian ini akan dibahas hasil pengujian terhadap perangkat lunak berdasarkan kriteria pengujian yang sudah ditentukan.
  \item[Bab VI Penutup] Pada bagian ini berisikan kesimpulan dan saran yang didapatkan dari hasil penelitian.
\end{description}

% Daftar buku atau karangan yang merupakan sumber rujukan dari sebuah
% tulisan atau karangan atau daftar tt suatu subjek ilmu, daftar pustaka
% Jumlah maksimal daftar pustaka dan referensi yang bisa dimasukkan adalah 100 item
\begin{thebibliography}{99}
\singlespacing 

% use the surname of the first author, followed by the last two digits of
% the year (hence lamport94)

\bibitem{challenging-issues-and-limitations-of-mobile-computing}
Deepak, G., and Dr. Pradeep B S. "Challenging Issues and Limitations of Mobile Computing."
  \emph{International Journal of Computer Technology and Applications} 3.1 (2012): Academic Journals Database. Web. 8 Jan. 2013.
  
  \bibitem {comparison-of-data-serialization-formats}
Audie Sumaray dan S. Kami Makki. "A comparison of data serialization formats for optimal efficiency on a mobile platform". \emph{6th International Conference on Ubiquitous Information Management and Communication} (2012): Artikel No. 48. ACM Digital Library. Web. 24 Jan 2013.
  
\bibitem{json-vs-xml-debate}
  \emph{Debate: JSON vs. XML as a data interchange format}
  http://www.infoq.com/news/2006/12/json-vs-xml-debate
  diakses pada 20 Januari 2012.
  
\bibitem{serialization-wikipedia}
  \emph{Serialization}
  http://en.wikipedia.org/wiki/Serialization
  diakses pada 24 Januari 2012.
  
\bibitem{agile-wikipedia}
  \emph{Agile software development}
  http://en.wikipedia.org/wiki/Agile\_software\_development
  diakses pada 24 Januari 2012.
  
\bibitem{json-fat-free}
  \emph{JSON: The Fat-Free Alternative to XML}
  http://www.json.org/xml.html
  diakses pada 20 Januari 2012.
  
\bibitem{web-api}
  \emph{Web API}
  http://en.wikipedia.org/wiki/Web\_API
  diakses pada 20 Januari 2012.

\bibitem{ws-restful}
  \emph{RESTful Web services: The basics}
  \\http://www.ibm.com/developerworks/webservices/library/ws-restful/
  diakses pada 14 September 2012.

\bibitem{introducing-json}
  \emph{Introducing JSON} http://www.json.org/
  diakses pada 20 Januari 2012.
    
\bibitem{rest-soap}
  \emph{How REST replaced SOAP on the Web: What it means to you}
  http://www.infoq.com/articles/rest-soap
  diakses pada 14 September 2012.
 
\end{thebibliography}

\end{document}

