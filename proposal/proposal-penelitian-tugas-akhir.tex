% Laporan Tugas Akhir
% Muhammad Ghazali - 0606036
\documentclass[a4paper, 12pt]{report}
\usepackage{graphicx, times}
\usepackage[bahasa]{babel}
\usepackage{tabularx}

\selectlanguage{bahasa}
%Gummi|063|=)
\title{\textbf{Web API CampusLife Berbasis REST untuk Menyediakan Layanan Permintaan Data}}
\author{
Muhammad Ghazali\\
Program Studi Teknik Informatika\\
Fakultas Teknik\\
Universitas Widyatama
\\ \texttt{<muhammadghazali2480@gmail.com>}
}
\date{\today}

\begin{document}

\maketitle

\tableofcontents
\setcounter{tocdepth}{3}

\listoffigures
\listoftables

% most research papers have an abstract, then there is a predefined commands
% for telling LaTeX which part of the content makes up the abstract. This
% should appear in its logical order, therefore, after the top matter, but
% before the main sections of the body.
\begin{abstract}
Abstract lorem ipsum is simply dummy text of the printing and typesetting industry. Lorem Ipsum has been the industry's standard dummy text ever since the 1500s, when an unknown printer took a galley of type and scrambled it to make a type specimen book. It has survived not only five centuries, but also the leap into electronic typesetting, remaining essentially unchanged. It was popularised in the 1960s with the release of Letraset sheets containing Lorem Ipsum passages, and more recently with desktop publishing software like Aldus PageMaker including versions of Lorem Ipsum.
\end{abstract}

\section*{Latar Belakang dan Masalah}
\addcontentsline{toc}{subsection}{Latar Belakang dan Masalah}
CampusLife adalah mobile information directory application yang dikembangkan oleh LayangLayang Mobile untuk menyediakan informasi yang relevan kepada civitas kampus. Untuk memenuhi kebutuhan penyediaan informasi kepada setiap penggunanya, maka LayangLayang Mobile perlu membuat Web API untuk membuka akses data agar setiap aplikasi CampusLife bisa menampilkan informasi yang dibutuhkan oleh penggunanya.

% TODO jelaskan metode-metode yang ada
% TODO jelaskan kelebihan dan kelemahan metode yang ada
% TODO jelaskan masalah pada metode yang dipilih
% TODO jelaskan solusi perbaikan metode
% TODO jelaskan rangkuman tujuan penelitian

% Pada bagian ini jelaskan rumusan masalah yang akan diteliti
\section*{Rumusan Masalah}
\addcontentsline{toc}{subsection}{Rumusan Masalah}
\begin{flushleft}
Masalah yang akan diangkat dalam penelitian ini adalah:
\begin{enumerate}
  \item Bagaimana implementasi web service berbasiskan cloud computing?
\end{enumerate}
\end{flushleft}

\section*{Batasan Masalah}
\addcontentsline{toc}{subsection}{Batasan Masalah}
\begin{flushleft}
\begin{itemize}
  \item Tidak membahas mengenai keamanan web service
  \item Pengembangan perangkat lunak menggunakan sebagian Scrum practice
\end{itemize}
\end{flushleft}

% Pada bagian tujuan, jelaskan tujuan pembangunan web service. Bagian ini berkaitan dengan rumusan masalah.
\section*{Tujuan}
\addcontentsline{toc}{subsection}{Tujuan}
\begin{flushleft}
Jelaskan tujuan penelitian
\end{flushleft}

% Pada bagian landasan teori, sertakan teori-teori yang berkaitan dangen 
% topik skripsi yang akan dikerjakan. Penyertaan teori dilakukan dengan
% mengutip dari beberapa referensi.
\section*{Landasan Teori}
\addcontentsline{toc}{subsection}{Landasan Teori}
\subsection*{RestFulrest}
\begin{flushleft}
TODO Jelaskan teori-teori yang mendasari penelitian ini.
\end{flushleft}

\section*{Metodologi Penelitian}
\addcontentsline{toc}{subsection}{Metodologi Penelitian}

% Buat tabel gant-chart untuk jadwal pengerjaan penelitian
\subsection*{Prosedur Penelitian}
\begin{flushleft}
Lorem ipsum.
\end{flushleft}

\subsection*{Metodologi Penelitian}
\begin{flushleft}
Lorem ipsum.
\end{flushleft}

% Buat tabel gant-chart untuk jadwal pengerjaan penelitian
\section*{Jadwal Kerja}
\addcontentsline{toc}{subsection}{Jadwal Kerja}
\begin{flushleft}
TODO Jadwal Kerja
\end{flushleft}

\section*{Usulan Pembimbing}
\addcontentsline{toc}{subsection}{Usulan Pembimbing}
Saya mengharapkan pembimbing yang benar-benar paham di bidang rekayasa perangkat lunak dan saya mengusulkan ibu Sriyani Violina untuk menjadi pembimbing pengerjaan tugas akhir saya.\cite{lamport94}

% Daftar buku atau karangan yang merupakan sumber rujukan dari sebuah
% tulisan atau karangan atau daftar tt suatu subjek ilmu, daftar pustaka
% Jumlah maksimal daftar pustaka dan referensi yang bisa dimasukkan adalah 100 item
\begin{thebibliography}{99}

% use the surname of the first author, followed by the last two digits of
% the year (hence lamport94)
\bibitem{rest-on-wikipedia}
  \emph{Representational state transfer}
  http://en.wikipedia.org/wiki/Representational\_state\_transfer
  diakses pada 14 September 2012

\bibitem{ws-restful}
  \emph{RESTful Web services: The basics}
  http://www.ibm.com/developerworks/webservices/library/ws-restful/
  diakses pada 14 September 2012
  
\bibitem{rest-soap}
  \emph{How REST replaced SOAP on the Web: What it means to you}
  http://www.infoq.com/articles/rest-soap
  diakses pada 14 September 2012
  
\bibitem{rest-soap-when-to-use-each}
  \emph{REST and SOAP: When Should I Use Each (or Both)?}
  http://www.infoq.com/articles/rest-soap-when-to-use-each
  diakses pada 14 September 2012
  
\bibitem{rest-and-now}
  \emph{REST, And Now for Something Completely Different}
  http://www.infoq.com/presentations/REST-And-Now-for-Something-Completely-Different
  diakses pada 14 September 2012
  
\bibitem{web-service-and-wikipedia}
  \emph{Web Service}
  http://en.wikipedia.org/wiki/Web\_service
  diakses pada 14 September 2012
  
\bibitem{resouce-oriended-architecture}
  \emph{Chapter 4: The Resource-Oriented Architecture}
  http://www.infoq.com/resource/articles/richardson-ruby-restful-w  s/enresources/04.pdf
  diakses pada 14 September 2012

\end{thebibliography}

\end{document}

