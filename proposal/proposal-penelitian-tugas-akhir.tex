% Proposal Skripsi
% Muhammad Ghazali - 0606036
\documentclass[a4paper, 12pt]{report}
\usepackage{setspace}
\usepackage{graphicx, times}
\usepackage[bahasa]{babel}
\usepackage{tabularx}
\usepackage[top=3cm, bottom=3cm, left=4cm, right=3cm]{geometry}

\selectlanguage{bahasa}
%Gummi|063|=)
\title{\textbf{Web API berbasis REST dengan menggunakan JSON sebagai format serialisasi data untuk meningkatkan efisiensi konsumsi data oleh aplikasi mobile CampusLife}}
\author{
Muhammad Ghazali\\
Program Studi Teknik Informatika\\
Fakultas Teknik\\
Universitas Widyatama
\\ \texttt{<muhammadghazali2480@gmail.com>}
}
\date{\today}

\begin{document}

\maketitle

\onehalfspacing
\tableofcontents
\setcounter{tocdepth}{3}

\listoffigures
\listoftables

% most research papers have an abstract, then there is a predefined commands
% for telling LaTeX which part of the content makes up the abstract. This
% should appear in its logical order, therefore, after the top matter, but
% before the main sections of the body.
\begin{abstract}
\onehalfspacing Abstract lorem ipsum is simply dummy text of the printing and typesetting industry. Lorem Ipsum has been the industry's standard dummy text ever since the 1500s, when an unknown printer took a galley of type and scrambled it to make a type specimen book. It has survived not only five centuries, but also the leap into electronic typesetting, remaining essentially unchanged. It was popularised in the 1960s with the release of Letraset sheets containing Lorem Ipsum passages, and more recently with desktop publishing software like Aldus PageMaker including versions of Lorem Ipsum.
\end{abstract}

% isi latar belakang dan masalah dibuat dengan mengikuti panduan:
% http://romisatriawahono.net/2012/06/18/kiat-menyusun-alur-latar-belakang-masalah-penelitian/
\section*{Latar Belakang dan Masalah}
\addcontentsline{toc}{subsection}{Latar Belakang dan Masalah}
% menjelaskan obyek penelitian
\onehalfspacing CampusLife adalah \textit{mobile information directory application} yang dikembangkan oleh LayangLayang Mobile\footnote{yang selanjutnya dalam proposal ini LayangLayang Mobile disebut sebagai LLM} untuk menyediakan informasi yang relevan kepada civitas kampus. Setiap informasi yang ada di CampusLife akan didistribusikan melalui Web API\footnote{http://en.wikipedia.org/wiki/Web\_API} ke seluruh pengguna \textit{smartphone} yang sudah memasang aplikasi CampusLife. Web API yang dibangun akan membuka akses ke \textit{resource}\footnote{http://en.wikipedia.org/wiki/Representational\_state\_transfer\#Guiding\_principles\_of\_the\_interface} yang disediakan oleh CampusLife, salah satunya Event\footnote{\textit{Resource} yang berisi detail informasi Event yang disediakan di CampusLife}. \textit{Resource} Event diolah berdasarkan data yang diambil dari \textit{data store}\footnote{http://en.wikipedia.org/wiki/Data\_store} yang tersimpan di salah satu layanan \textit{Database as a Service}\footnote{http://en.wikipedia.org/wiki/Cloud\_database} yang digunakan oleh LLM di AppFog\footnote{http://www.appfog.com/}. \textit{Resource} Event yang sudah diolah tersebut akan dikonsumsi oleh Web API \textit{consumer} seperti perambah web atau \textit{mobile application} dalam format serialisasi data tertentu.

\onehalfspacing \textit{Smartphone} merupakan salah satu \textit{mobile computing devices} yang memiliki masa hidup baterai dan ketersediaan \textit{bandwidth} yang terbatas.\cite{challenging-issues-and-limitations-of-mobile-computing} Dengan kedua keterbatasan tersebut proses pertukaran data antara Web API dan \textit{mobile application} perlu dilakukan dengan efisien dalam hal ukuran data yang dikirimkan antara Web API dan \textit{mobile application}.

\onehalfspacing Sampai awal tahun 2013 ini, sudah ada lebih dari 10 format serialisasi data\footnote{http://en.wikipedia.org/wiki/Comparison\_of\_data\_serialization\_formats}. Pemilihan format serialisasi data harus dilakukan dengan tepat agar ukuran data yang dikirimkan antara Web API dan \textit{mobile application} bisa memiliki ukuran data yang kecil. Dalam penelitian ini penulis hanya akan mengambil 2 format serialisasi data untuk dibandingkan, yaitu JSON\footnote{http://json.org/} dan XML\footnote{http://www.w3.org/TR/REC-xml/}. Penulis memilih kedua format serialisasi data tersebut karena tertarik untuk mengetahui langsung apakah JSON merupakan alternatif yang tepat dibandingkan XML. Selain itu penulis tertarik untuk mempelajari lebih lanjut kedua format serialisasi data tersebut karena banyak perdebatan yang membandingkan antara JSON dan XML. \cite{json-vs-xml-debate}

% menjelaskan rangkuman penelitian
\onehalfspacing Dalam penelitian ini penulis akan mengkaji penerapan format serialisasi data JSON untuk digunakan dalam proses pertukaran data antara Web API dan CampusLife \textit{mobile application}. Hasil akhir yang diharapkan adalah format serialisasi data JSON yang digunakan mampu memberikan ukuran \textit{resource}\footnote{http://en.wikipedia.org/wiki/Resource\_(Web)} yang lebih kecil dibandingkan format serialisasi data XML.\cite{json-fat-free}

% Pada bagian ini jelaskan rumusan masalah yang akan diteliti
\section*{Rumusan Masalah}
\addcontentsline{toc}{subsection}{Rumusan Masalah}
\onehalfspacing Berdasarkan latar belakang masalah yang telah dipaparkan, maka masalah yang akan diteliti dalam penelitian ini dapat dirumuskan sebagai berikut:
\begin{enumerate}
  \item Seberapa kecil ukuran data yang dapat dikirimkan apabila format serialisasi data yang digunakan adalah JSON?
  \item Seberapa kecil ukuran data yang dapat dikirimkan apabila format serialisasi data yang digunakan adalah XML?
\end{enumerate}

\section*{Batasan Masalah}
\addcontentsline{toc}{subsection}{Batasan Masalah}
\onehalfspacing Untuk memperjelas ruang lingkup pelaksanaan penelitian, penulis memiliki batasan masalah meliputi:
\onehalfspacing
\begin{itemize}
  \item Tidak membahas mengenai keamanan perangkat lunak, data dan jaringan.
  \item Pengembangan perangkat lunak menggunakan sebagian praktek dari Scrum dan tidak akan membahas Scrum secara komprehensif.
  \item Pengembangan Web API hanya akan tahap purwa-rupa.
  \item Pengembangan Web API hanya akan meliputi proses pengambilan data \textit{event}.
  \item Skema data \textit{event} sudah disediakan akan disediakan oleh pihak LayangLayang Mobile.
  \item Jika situasi dan kondisi memungkinkan, penulis akan melakukan demo untuk mengakses Web API melalui smartphone yang menggunakan sistem operasi Android.
\end{itemize}

% Pada bagian tujuan, jelaskan tujuan pembangunan web service. Bagian ini berkaitan dengan rumusan masalah.
\section*{Tujuan}
\addcontentsline{toc}{subsection}{Tujuan}
\onehalfspacing 
Menerapkan format serialisasi data yang optimal pada pengiriman data antara Web API dan \textit{mobile application}, sehingga dapat menghasilkan ukuran data yang efisien.

% Pada bagian landasan teori, sertakan teori-teori yang berkaitan dangen 
% topik skripsi yang akan dikerjakan. Penyertaan teori dilakukan dengan
% mengutip dari beberapa referensi.
\section*{Landasan Teori}
\addcontentsline{toc}{subsection}{Landasan Teori}
\onehalfspacing
\subsection*{RestFulrest}
\begin{flushleft}
TODO Jelaskan teori-teori yang mendasari penelitian ini.
\end{flushleft}

\section*{Metodologi Penelitian}
\addcontentsline{toc}{subsection}{Metodologi Penelitian}

% Buat tabel gant-chart untuk jadwal pengerjaan penelitian
\subsection*{Prosedur Penelitian}
\begin{flushleft}
Lorem ipsum.
\end{flushleft}

\subsection*{Metodologi Penelitian}
\begin{flushleft}
Lorem ipsum.
\end{flushleft}

% Buat tabel gant-chart untuk jadwal pengerjaan penelitian
\section*{Jadwal Kerja}
\addcontentsline{toc}{subsection}{Jadwal Kerja}
\onehalfspacing 
\begin{flushleft}
TODO Jadwal Kerja
\end{flushleft}

\section*{Usulan Pembimbing}
\addcontentsline{toc}{subsection}{Usulan Pembimbing}
\onehalfspacing Saya mengharapkan pembimbing yang benar-benar paham di bidang rekayasa perangkat lunak dan saya mengusulkan ibu Sriyani Violina untuk menjadi pembimbing pengerjaan tugas akhir saya.

% Daftar buku atau karangan yang merupakan sumber rujukan dari sebuah
% tulisan atau karangan atau daftar tt suatu subjek ilmu, daftar pustaka
% Jumlah maksimal daftar pustaka dan referensi yang bisa dimasukkan adalah 100 item
\begin{thebibliography}{99}
\singlespacing 

% use the surname of the first author, followed by the last two digits of
% the year (hence lamport94)

\bibitem{challenging-issues-and-limitations-of-mobile-computing}
Deepak, G., and Dr. Pradeep B S. "Challenging Issues and Limitations of Mobile Computing." \emph{International Journal of Computer Technology and Applications} 3.1 (2012): Academic Journals Database. Web. 8 Jan. 2013.

\bibitem{programmableweb-apis}
  \emph{API Dashboard}
  http://www.programmableweb.com/apis
  diakses pada 27 Nopember 2012
  
\bibitem{json-vs-xml-debate}
  \emph{Debate: JSON vs. XML as a data interchange format}
  http://www.infoq.com/news/2006/12/json-vs-xml-debate
  diakses pada 20 Januari 2012
  
\bibitem{json-fat-free}
  \emph{JSON: The Fat-Free Alternative to XML}
  http://www.json.org/xml.html
  diakses pada 20 Januari 2012

\bibitem{rest-on-wikipedia}
  \emph{Representational state transfer}
  http://en.wikipedia.org/\\wiki/Representational\_state\_transfer
  diakses pada 14 September 2012 

\bibitem{ws-restful}
  \emph{RESTful Web services: The basics}
  http://www.ibm.com/\\developerworks/webservices/library/ws-restful/
  diakses pada 14 September 2012
  
\bibitem{rest-soap}
  \emph{How REST replaced SOAP on the Web: What it means to you}
  http://www.infoq.com/articles/rest-soap
  diakses pada 14 September 2012
  
\bibitem{rest-soap-when-to-use-each}
  \emph{REST and SOAP: When Should I Use Each (or Both)?}
  http://www.infoq.com/articles/rest-soap-when-to-use-each
  diakses pada 14 September 2012
  
\bibitem{rest-and-now}
  \emph{REST, And Now for Something Completely Different}
  http://www.infoq.com/presentations/REST-And-Now-for-Something-Completely-Different
  diakses pada 14 September 2012
 
\end{thebibliography}

\end{document}

