\documentclass[landscape]{article}
\usepackage[margin=1in]{geometry}
\usepackage{verbatim,multicol}

\title{Latex for Linguists: Plain Paper}
\author{Paul M.~Heider\\pmheider@buffalo.edu}
\date{} %% 24 January 2011

\begin{document}
\begin{multicols}{2}
\maketitle
\section{The content vs. formatting distinction}

\subsection*{Content:}
\begin{verbatim}
       Latex for Linguists Outline 
        Paul M. Heider  pmheider@buffalo.edu 
\end{verbatim}

\subsection*{Formatting:}
\begin{verbatim}
\title{                           }
\author{      .~      \\                    }
\end{verbatim}

\subsection*{Content:}
\begin{verbatim}
      Things you will         not  learn

          How to install LaTeX
\end{verbatim}

\subsection*{Formatting:}
\begin{verbatim}
\item                 \textbf{   }      
  \begin{itemize}
    \item 
  \end{itemize}
\end{verbatim}

\columnbreak

\verbatiminput{latex_for_linguists_plain_paper_pg1.tex}

\pagebreak

\section{Sections of a document}
\subsection{Front matter}
This is where you put all your document-general declarations (e.g., packages, commands, variables).

\begin{verbatim}
\documentclass[12pt]{article}

\usepackage{geometry}

\title{Latex for Linguists: Plain Paper}
\author{Paul M.~Heider\\pmheider@buffalo.edu}
\date{} %% 24 January 2011

\begin{document}
\end{verbatim}
\subsection{Back matter}
Nothing after the end of the document is parsed.  
I use that space to store old notes, reminders, etc.

\begin{verbatim}
\end{document}
...
\end{verbatim}
\subsection{And everything in between}
\begin{itemize}
  \item section, subsection, subsubsection
  \item section*, subsection*, subsubsection*
  \item paragraph, subparagraph
  \item appendix %% (does what you might expect)
  \item chapter, part %% (used by other document classes)
\end{itemize}

\paragraph{Summary} I only recently started using \verb=\paragraph= to match JML journal styling.
\subparagraph{Dirty Secret} I've never used \verb=\subparagraph=.
I didn't even know about it until Saturday.

\columnbreak

\verbatiminput{latex_for_linguists_plain_paper_pg2.tex}

\pagebreak

\section{Basic formatting options}
\subsection{Typefacing}
\begin{description}
  \item[textbf] All the item description are \textbf{bolded} by \textbf{default}.
  \item[textsl] Sometimes, you need \textsl{italics}.
  \item[textsc] Semanticists like to put concepts (e.g., \textsc{Truth} and \textsc{Bank}) in \textsc{SmallCaps}
  \item[texttt] Computational linguists sometimes need to show their underlying \texttt{000110} ('\texttt{code}').
\end{description}

\subsection{Font size}
Mostly, you shouldn't mess with font size (cf., a combination of Mantras 1 and 3).
If you \textsl{need} to:
\begin{center}
  \tiny{tiny}, \scriptsize{scriptsize}, \footnotesize{footnotesize}, \normalsize{normalsize}, \large{large}, \Large{Large}, \huge{huge}
\end{center}

%% And don't forget to reset it later
\normalsize 
\subsection{Margins}
Remember the \texttt{geometry} package that keeps showing up in the header?
\begin{verbatim}
\usepackage[margin=1in]{geometry}
\usepackage[left=2in,top=0.75in,right=4in]{geometry}
... %% (RTFM)
\end{verbatim}

It's not the only way, of course.  
If you don't like \texttt{geometry}, \texttt{fullpage} is probably the next easiest.

\subsection{Spacing} %% Maybe mention periods, ~, and :?
By default, {\LaTeX} keeps linespacing at 1.  
Now, if you want to see what happens when there are larger spaces between lines, then you'll want to use the \verb=\linespread{X}= command.

\linespread{1.67} %% linespread changes the spacing
\selectfont %% selectfont puts it into effect

When you look at the code on the right, you'll notice that I didn't use ``2'' as my measure.
Also, \verb=\linespread{1.3}= gets you 1.5-spacing.

\linespread{1}\selectfont

\columnbreak

\small
\verbatiminput{latex_for_linguists_plain_paper_pg3.tex}
\normalsize

\end{multicols}
\end{document}

\section{Common mistakes}
\begin{itemize}
  \item single and double quotes
  \item ampersands
  \item not closing a table or gb4e example
\end{itemize}
